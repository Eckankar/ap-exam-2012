\documentclass[11pt,a4paper]{article}

\usepackage[utf8]{inputenc}
\usepackage[english]{babel}
\usepackage[T1]{fontenc}

\usepackage{amsmath,amssymb,amsfonts}

\usepackage{hyperref}
\usepackage{minted}

\title{Exam in Advanced Programming, 2012}
\author{Sebastian Paaske Tørholm}

\newcommand{\hoogle}[1]{\href{http://www.haskell.org/hoogle/?hoogle=#1}
                             {\texttt{#1}}}

\begin{document}
\maketitle

\section{IVars in Erlang}
The IVars are modelled using Erlang processes. Each IVar starts in an empty
state, changing to a set state once a value has been specified.

Immutability is ensured by throwing away \texttt{put}-requests when in the set
state, marking the ivar as compromised if required by the specification.

For the princess variables, robustness with regards to malformed predicates in
handled by evaluating the predicate inside a \texttt{try} clause; ignoring the
\texttt{put} request if anything but \texttt{true} is returned, or an exception
of any kind is thrown.

The functions in \texttt{pmuse} have been solved using a simple map-reduce
implementation. Each mapper is spawned with a mapping function, which is given
pairs of data and an IVar to store the result in. This makes it possible
to use the same mapper for both \texttt{pmmap} and \texttt{treeforall}.
Distribution of problems is done in the same way that the map reduce framework
from assignment 3 does it. Reduction is done in the original thread, and is
thus not parallellized in any way.

Order is maintained in \texttt{pmmap} by keeping a list of \texttt{IVars} in
the correct order. These can then be queried for the computed values upon
reducing. Values are only \texttt{put} into each IVar once in one mapping
task, so no IVar will be compromised.

Using princess predicates for \texttt{treeforall} is slightly awkward.
If we perform a \texttt{get} on an \texttt{IVar} for which the predicate
rejected the entry, we block the thread. This problem is resolved by using
a custom predicate, which uses the supplied predicate on the data contained
in values of the form \texttt{\{ok, D\}}, and always accepting values of the
form \texttt{notok}. We then put the value we wish to check, wrapped in a
\texttt{\{ok,D\}} first, followed by putting a \texttt{notok}. This guarantees
us that a value will eventually be accepted. Since the order of messages
is maintained when sending from one process to another, we know that the
\texttt{ok}-message will be processed first. This presents no problem with
regards to compromising the IVar, as princess IVars cannot be compromised.

An exception is used to break out of the reducer as soon as a falsy value is
found. As such, we terminate our reduction at the earliest possible point.

\subsection{Assessment}
The code tries to be indented well, using meaningful variable- and function
names. Pattern matching is used whenever it makes sense. The supplied unit
testing provides a good indication that the code in all likelihood satisfies
the specification.

Using \texttt{pmmap} or \texttt{treeforall} on large lists (of 100,000
elements, for instance) may be problematic, since a process is spawned for
each IVar, and an IVar is created for each element in the list.

\section{Parsing Soil}
The parsing is done using \hoogle{ReadP}.

The structure of the parser construction mostly follows the structure
suggested by the grammar. Instead of having two rules each for parameter lists
and argument lists, these rules have been simplified using \hoogle{sepBy}.

\emph{Prim} had a problem with left recursion in the rule for \texttt{concat}. This
problem was resolved by special-casing \texttt{concat} using \hoogle{chainl1}, and
moving the remaining rules into the nonterminal \emph{PrimBasic}.

In order to allow for arbitrary whitespace, \hoogle{skipSpaces} is used whereever
whitespace is allowed. In general, the parsers only handle whitespace in front of
them, leaving the \texttt{program} parser to deal with trailing whitespace.
To simplify this process, helper parsers \texttt{schar} and \texttt{sstring} have
been introduced, as as \hoogle{char} and \hoogle{string}, that ignores arbitrary
preceding whitespace.

I have chosen to use a datatype to represent parse errors. The datatype is capable
of representing the two types of error that can occur from parsing:
\begin{itemize}
    \item There are multiple ways to interpret a given source text.
    \item The source text doesn't represent a valid program.
\end{itemize}

I'm not quite sure if the first of these can occur. It may be possible due to
arbitrary\footnote{And thus, also 0.} whitespace being allowed between tokens.

\subsection{Assessment}
The parser has been segmented into sub-parsers, each one of a managable size,
making it easier to inspect them for correctness. Combinators of varying
types are used from \hoogle{ReadP}, rather than reinventing them. This yields
shorter, more readable code. The code attempts to embrace Haskell's style, and
has been run through HLint and ghc-mod check without any notices.

The code comes with unit tests. These unit tests verify the given example
programs against their expected parse tree. The also try to verify correct
response to failure, and proper associativity of \texttt{concat}. The tests
would suggest that the parser works correctly.

Further correctness checking could be done using quickcheck, by generating an
AST, serializing it to source code, re-parsing it and comparing the result to
the original AST.

\section{Interpreting Soil}
The name and function environments are modelled using simple lists of tuples.
This means we get a $O(n)$ lookup time, which could be improved by changing the
datatype if necessary.

The process type has been expanded to contain a process identifier, making the
process queue a simple list. The laziness of the lists provides for asymptotically
constant concatenation, making them perform well for the necessary operations.

The program is evaluated in the \texttt{ProgramEvaluation} monad, which
maintains a program state consisting of the function environment and process
queue. The function environment is immutable, which is reflected in the type
of \texttt{runPE}.

Several helpful monadic values have been made, making it possible to write
relatively clean monadic code.

I have chosen to handle static errors using the \hoogle{error} function.
This makes it easy to verify that no such error occurs in testing, and is a
simple, lightweight and local solution to handling those error conditions
cleanly. In addition to the static errors provided in the problem text, a
few extra ones have been added. \texttt{primToName} throws an error if a
non-name primitive occur in the syntax tree, where such primitives shouldn't
be allowed. In addition, if there at any point is no process in the process
queue when doing a round robin, \footnote{Which should be an impossibility,
as there always should exist at least two processes, \texttt{\#println} and
\texttt{\#errorlog}.} an error is also thrown. All other errors are handled by
sending a suitable message to the \texttt{\#errorlog} process, and ignoring
whatever operation fails. This gives the required

The processes \texttt{\#println} and \texttt{\#errorlog} are special-cased,
but still exist in the process table. All values except for their mailbox and
process id are dummy values.

Evaluating all possible outcomes is done without using a monad transformer,
though it seems like it'd likely candidate to be rewritten to use one. In the
same vein, the \texttt{ProgramEvaluation} monad could have been written using
the state monad, as that essentially is what it is.

\subsection{Assessment}
The code style could most likely be improved; a lot of places checking has to
be done to ensure that the input is correct. This leads to much unpacking of
option types, verification of length of lists and so on. The monadic style
does help simplify the transfer and maintenance of state, making it it simpler
to work with.

The code runs through \texttt{ghc-mod check} and \texttt{hlint} without any
notices or warnings.

Basic correctness checks are done by running the interpreter on the sample
programs provided, as well as doing manual evaluation. Further correctness
checking is a good idea, as the interpreter is rather complex, and subtle bugs
easily could hide in the corners. To improve correctness checking, further
programs could be written. In addition, quickcheck could be used to compare
output to a reference implementation to further catch any bugs.

\appendix
\section{Code}
\subsection{IVars in Erlang}
\subsubsection{pm.erl}
\inputminted{erlang}{src/pm/pm.erl}
\subsubsection{pmuse.erl}
\inputminted{erlang}{src/pm/pmuse.erl}
\subsubsection{pmtest.erl}
\inputminted{erlang}{src/pm/pmtest.erl}
\subsubsection{pmusetest.erl}
\inputminted{erlang}{src/pm/pmusetest.erl}

\subsection{Parsing Soil}
\subsubsection{SoilParser.hs}
\inputminted{haskell}{src/soil/SoilParser.hs}
\subsubsection{SoilParserTest.hs}
\inputminted{haskell}{src/soil/SoilParserTest.hs}

\subsection{Interpreting Soil}
\subsubsection{SoilInterp.hs}
\inputminted{haskell}{src/soil/SoilInterp.hs}
\subsubsection{SoilInterpTest.hs}
\inputminted{haskell}{src/soil/SoilInterpTest.hs}
\end{document}

