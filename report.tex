\documentclass[11pt,a4paper]{article}

\usepackage[utf8]{inputenc}
\usepackage[english]{babel}
\usepackage[T1]{fontenc}

\usepackage{amsmath,amssymb,amsfonts}

\usepackage{minted}

\title{Exam in Advanced Programming, 2012}
\author{Sebastian Paaske Tørholm}

\begin{document}
\maketitle

\section{IVars in Erlang}
The IVars are modelled using Erlang processes. Each IVar starts in an empty
state, changing to a set state once a value has been specified.

Immutability is ensured by throwing away \texttt{put}-requests when in the set
state, marking the ivar as compromised if required by the specification.

For the princess variables, robustness with regards to malformed predicates in
handled by evaluating the predicate inside a \texttt{try} clause; ignoring the
\texttt{put} request if anything but \texttt{true} is returned, or an exception
of any kind is thrown.

The functions in \texttt{pmuse} have been solved using a simple map-reduce
implementation. Each mapper is spawned with a mapping function, which is given
pairs of data and an IVar to store the result in. This makes it possible
to use the same mapper for both \texttt{pmmap} and \texttt{treeforall}.
Distribution of problems is done in the same way that the map reduce framework
from assignment 3 does it. Reduction is done in the original thread, and is
thus not parallellized in any way.

Order is maintained in \texttt{pmmap} by keeping a list of \texttt{IVars} in
the correct order. These can then be queried for the computed values upon
reducing. Values are only \texttt{put} into each IVar once in one mapping
task, so no IVar will be compromised.

Using princess predicates for \texttt{treeforall} is slightly awkward.
If we perform a \texttt{get} on an \texttt{IVar} for which the predicate
rejected the entry, we block the thread. This problem is resolved by using
a custom predicate, which uses the supplied predicate on the data contained
in values of the form \texttt{\{ok, D\}}, and always accepting values of the
form \texttt{notok}. We then put the value we wish to check, wrapped in a
\texttt{\{ok,D\}} first, followed by putting a \texttt{notok}. This guarantees
us that a value will eventually be accepted. Since the order of messages
is maintained when sending from one process to another, we know that the
\texttt{ok}-message will be processed first. This presents no problem with
regards to compromising the IVar, as princess IVars cannot be compromised.

An exception is used to break out of the reducer as soon as a falsy value is
found. As such, we terminate our reduction at the earliest possible point.

\subsection{Assessment}
The code tries to be indented well, using meaningful variable- and function
names. Pattern matching is used whenever it makes sense. The supplied unit
testing provides a good indication that the code in all likelihood satisfies
the specification.

Using \texttt{pmmap} or \texttt{treeforall} on large lists (of 100,000
elements, for instance) may be problematic, since a process is spawned for
each IVar, and an IVar is created for each element in the list.

\appendix
\section{Code}
\subsection{IVars in Erlang}
\subsubsection{pm.erl}
\inputminted{erlang}{src/pm/pm.erl}
\subsubsection{pmuse.erl}
\inputminted{erlang}{src/pm/pmuse.erl}
\subsubsection{pmtest.erl}
\inputminted{erlang}{src/pm/pmtest.erl}
\subsubsection{pmusetest.erl}
\inputminted{erlang}{src/pm/pmusetest.erl}

\end{document}

